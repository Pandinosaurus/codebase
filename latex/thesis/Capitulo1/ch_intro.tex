%%%%%%%%%%%%%%%%%%%%%%%%%%%%%%%%%%%%%%%%%%%%%%%%%%%%%%%%%%%%%%%%%%%%%%%%%%%%%%%%
%2345678901234567890123456789012345678901234567890123456789012345678901234567890
%        1         2         3         4         5         6         7         8
%%%%%%%%%%%%%%%%%%%%%%%%%%%%%%%%%%%%%%%%%%%%%%%%%%%%%%%%%%%%%%%%%%%%%%%%%%%%%%%%
\chapter{Introduction} \label{ch_intro}
\textit{This chapter deals with the initial introduction, motivation and presentation  of this thesis. The thesis addresses aspects related to planning and control of complex postural tasks for humanoid robots using human learning by demonstration. The thesis proposes a specific situation that has to be solved by the robot. It is the order \robotorder that represents a complex task which involves the development of several skills and gives rise to a broad set of approaches to solve this problem. The present work is focused on discussing and studying methods that involves the generation and execution of humanoid motions in terms of postural body transitions. These postural body transitions are optimized using an index that we called the reward profile. The reward profile  is a multi-modal time-dependent function that encodes the skill goal that the robot has to perform. At the same time, it is a measurement of the skill performance in terms of different aspects like stability, softness or human likeliness. Furthermore, this thesis addresses the modelling and control of the humanoid robot and presents a wide variety of simulated and experimental results. The objective behind this work is to make humanoid robots more intelligent and autonomous, by allowing them to follow complex orders safely and precisely.}
\newpage
%%%%%%%%%%%%%%%%%%%%%%%%%%%%%%%%%%%%%%%%%%%%%%%%%%%%%%%%%%%%%%%%%%%%%%%%%%%%%%%%
%\input{Capitulo1/motivation}
%\input{Capitulo1/objectives}
%\input{Capitulo1/organization}
%%%%%%%%%%%%%%%%%%%%%%%%%%%%%%%%%%%%%%%%%%%%%%%%%%%%%%%%%%%%%%%%%%%%%%%%%%%%%%%%








