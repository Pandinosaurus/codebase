%%%%%%%%%%%%%%%%%%%%%%%%%%%%%%%%%%%%%%%%%%%%%%%%%%%%%%%%%%%%%%%%%%%%%%%%%%%%%%%%
%2345678901234567890123456789012345678901234567890123456789012345678901234567890
%        1         2         3         4         5         6         7         8
\chapter{Basic Representations for Postural Control in Humanoids}\label{ch_basics}
\textit{This chapter deals with some basic concepts that are repeatably used in this thesis. There are some basic tasks that have to be done in advance in order to conduct many of the experiments. They are related to modeling, kinematics, control and dynamics. There are many aspects of the humanoid set up that are not explicitly described in the thesis. This chapter acts as the basis of some of the algorithms that are explained in the next chapters. First, a study of different humanoid robot models is presented. It includes simple models to represent the robot, like the inverted pendulum, and complex models like the mass distributed model. The equation of motion of each of them is obtained. These models are included in some  motion generation algorithms and balance maintenance methods. Afterwards, the most famous stability criterion, the Zero Moment Point (ZMP),  is explained in detail. Furthermore, two common methods of biped locomotion generation are studied, the 3D Linear Inverted Pendulum Model and the Cart Table model. The latter was used to generate stable biped locomotion in the real humanoid. Finally, a simple method of whole body imitation is presented and a dance performance is obtained as an experiment. }
\newpage
%%%%%%%%%%%%%%%%%%%%%%%%%%%%%%%%%%%%%%%%%%%%%%%%%%%%%%%%%%%%%%%%%%%%%%%%%%%%%%%%
\input{Capitulo2/intro}
\input{Capitulo2/model}
\input{Capitulo2/generation}
\input{Capitulo2/dancing}
\input{Capitulo2/conclusions}
