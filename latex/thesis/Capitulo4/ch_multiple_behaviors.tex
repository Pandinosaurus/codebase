%%%%%%%%%%%%%%%%%%%%%%%%%%%%%%%%%%%%%%%%%%%%%%%%%%%%%%%%%%%%%%%%%%%%%%%%%%%%%%%%
%2345678901234567890123456789012345678901234567890123456789012345678901234567890
%        1         2         3         4         5         6         7         8
%%%%%%%%%%%%%%%%%%%%%%%%%%%%%%%%%%%%%%%%%%%%%%%%%%%%%%%%%%%%%%%%%%%%%%%%%%%%%%%%
\chapter{Learning and Improving a Sequence of Goal Directed Skills}\label{ch_multiple_behaviors}
\textit{As it was discussed in the previous chapter, there are evidences that justify that the imitation between humans are goal-directed. We proposed there a new method to acquire a single skill from human demonstrations. However, it is quite common for a human being to perform several skills sequentially, for example, to walk to a door and open it. Therefore, when performing multiple skills, we internally define an unknown optimal policy to satisfy multiple goals. This chapter presents a method to transfer a complex behavior composed by multiple skills  from a human demonstrator to a humanoid robot. We defined a multi-objective reward function as a measurement of the goal optimality for both human and robot, which is defined in each subtask of the global behavior. We optimized a hierarchical policy to generate whole-body movements for the robot that produces a reward profile that is compared and matched with the human reward profile, producing an imitative behavior. Furthermore, we can search in the proximity of the solution space to improve the reward profile and innovate a new solution, which is more beneficial for the humanoid. Experiments were carried out in a real  humanoid robot.}
\newpage
%%%%%%%%%%%%%%%%%%%%%%%%%%%%%%%%%%%%%%%%%%%%%%%%%%%%%%%%%%%%%%%%%%%%%%%%%%%%%%%%
% chapter1
%\input{Capitulo4/intro}
%\input{Capitulo4/related_work}
%\input{Capitulo4/open_door}
%chapter2
%\input{Capitulo4/policy}
%\input{Capitulo4/gmm_gmr}
%\input{Capitulo4/primitives}
%\input{Capitulo4/policy_optimization}
%chapter3
%\input{Capitulo4/demonstrations}
%\input{Capitulo4/behavior_selector}
%\input{Capitulo4/reward}
%\input{Capitulo4/trajectory_generation}
%chapter4
%\input{Capitulo4/conclusions}
