%%%%%%%%%%%%%%%%%%%%%%%%%%%%%%%%%%%%%%%%%%%%%%%%%%%%%%%%%%%%%%%%%%%%%%%%%%%%%%%%
%2345678901234567890123456789012345678901234567890123456789012345678901234567890
%        1         2         3         4         5         6         7         8
%%%%%%%%%%%%%%%%%%%%%%%%%%%%%%%%%%%%%%%%%%%%%%%%%%%%%%%%%%%%%%%%%%%%%%%%%%%%%%%%
\chapter{Robust Control of Humanoid Models through Fractional Calculus}\label{frac_chapter}
\textit{There is an open discussion between those who defend mass distributed models for humanoid robots and those in favor of simple concentrated models. Even though each of them has its advantages and disadvantages, little research has been conducted analyzing the control performance due to the mismatch between the model and the real robot, and how the simplifications affect the controller’s output. In this chapter we address this problem by combining a reduced model of the humanoid robot, which has an easy mathematical formulation and implementation, with a fractional order controller, which is robust to changes in the model parameters. This controller is a generalization of the well-known PID structure obtained from the application of Fractional Calculus to control. This control strategy guarantees the robustness of the system, minimizing the effects from the assumption that the robot has a simple mass distribution. The humanoid robot is modeled and identified as a triple inverted pendulum and, using a gain scheduling strategy, the performances of a classical PID controller and a fractional order PID controller are compared, tuning the controller parameters with a genetic algorithm.}
\newpage
%%%%%%%%%%%%%%%%%%%%%%%%%%%%%%%%%%%%%%%%%%%%%%%%%%%%%%%%%%%%%%%%%%%%%%%%%%%%%%%%
\input{Capitulo5/intro}
\input{Capitulo5/related_work}
\input{Capitulo5/fractional_controlers}
\input{Capitulo5/humanoid_robust_control}
\input{Capitulo5/results}
\input{Capitulo5/conclusions}



