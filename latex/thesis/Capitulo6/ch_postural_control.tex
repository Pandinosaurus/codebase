%%%%%%%%%%%%%%%%%%%%%%%%%%%%%%%%%%%%%%%%%%%%%%%%%%%%%%%%%%%%%%%%%%%%%%%%%%%%%%%%
%2345678901234567890123456789012345678901234567890123456789012345678901234567890
%        1         2         3         4         5         6         7         8
%%%%%%%%%%%%%%%%%%%%%%%%%%%%%%%%%%%%%%%%%%%%%%%%%%%%%%%%%%%%%%%%%%%%%%%%%%%%%%%%
%\chapter{Postural Planning and Control in Complex Environments}\label{ch_postural_control}
\chapter{Control of Humanoid Robots Executing Complex Tasks}\label{ch_postural_control}
\textit{This chapter deals with the planning and execution of a complex task ordered to a humanoid robot. The robot has to be able to execute high level postural tasks in the presence of a cluttered environment. The motion execution must be soft and stable and, at the same time, the robot has to be able to successfully avoid obstacles in the environment. First, the robot has to identify the environment and the obstacles. Second, the robot has to be able to move from the initial point to the final point performing a set of postural movements. The postural task is performed in two levels, a postural planning, which off-line computes  the  safe and stable postural movement that allows the robot to  navigate through the environment, and an online postural control, which has to do with the execution, control and disturbance rejection that makes possible the fulfillment of the task. This chapter encompasses the learning strategies explained in chapters \ref{ch_imitation} and \ref{ch_multiple_behaviors}, the control method of chapter \ref{frac_chapter}, while using methodologies of chapter \ref{ch_basics}. The task selected as an example is a robot that starts seated on a chair, stands up, walks avoiding obstacles until it reaches a door, opens the door and leaves the room. This chapter also gives a practical significance to this thesis. }
\newpage
%%%%%%%%%%%%%%%%%%%%%%%%%%%%%%%%%%%%%%%%%%%%%%%%%%%%%%%%%%%%%%%%%%%%%%%%%%%%%%%%

\input{Capitulo6/intro}
\input{Capitulo6/environment}
\input{Capitulo6/postural_planning}
\input{Capitulo6/postural_control}
\input{Capitulo6/results}
\input{Capitulo6/conclusions}
