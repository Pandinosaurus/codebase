%%%%%%%%%%%%%%%%%%%%%%%%%%%%%%%%%%%%%%%%%%%%%%%%%%%%%%%%%%%%%%%%%%%%%%%%%%%%%%%%
%2345678901234567890123456789012345678901234567890123456789012345678901234567890
%        1         2         3         4         5         6         7         8
%%%%%%%%%%%%%%%%%%%%%%%%%%%%%%%%%%%%%%%%%%%%%%%%%%%%%%%%%%%%%%%%%%%%%%%%%%%%%%%%
\chapter{Conclusions and Future Works}\label{ch_conclusions}
\textit{ This thesis has attempted to contribute in some areas related to  posture behavior of humanoid robots.  From the beginning, this work has had a differential feature if it is compared with a usual Ph.D. thesis. It has an initial framework which is the center of all  developed studies. The framework is the high level order  \robotorder that the robot needs to acomplish. To execute that order, the humanoid robot needs to perform a set of  movements, it needs to avoid a series of obstacles and it needs to adopt a determined posture or set of postures. All needed procedures to successfully achieve this order are gathered in this thesis. The amount of contributions that this thesis has generated are related to learning from demonstration, reinforcement learning, non-linear control and motion planning. In this chapter the conclusions of this thesis are summarized and a discussion of the positive  as well as the negative aspects of the developed work is presented. There are also proposed some future lines of development that can be used to improve the current work or to serve as a start point for new developments. }
\newpage
%%%%%%%%%%%%%%%%%%%%%%%%%%%%%%%%%%%%%%%%%%%%%%%%%%%%%%%%%%%%%%%%%%%%%%%%%%%%%%%%
\input{Capitulo7/conclusions}
\input{Capitulo7/contributions}
\input{Capitulo7/future}
\input{Capitulo7/publications}






