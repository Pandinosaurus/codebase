%%%%%%%%%%%%%%%%%%%%%%%%%%%%%%%%%%%%%%%%%%%%%%%%%%%%%%%%%%%%%%%%%%%%%%%%%%%%%%%%
%2345678901234567890123456789012345678901234567890123456789012345678901234567890
%        1         2         3         4         5         6         7         8
%%%%%%%%%%%%%%%%%%%%%%%%%%%%%%%%%%%%%%%%%%%%%%%%%%%%%%%%%%%%%%%%%%%%%%%%%%%%%%%%
%\vspace{-2cm}
Esta tesis aborda el problema de la planificación y control de tareas complejas de un robot humanoide desde el punto de vista postural. Viene motivada por el auge  de la robótica en la sociedad actual, donde ya se están incorporando robots sencillos y su objetivo es avanzar en el desarrollo de comportamientos complejos en robots humanoides, para que en el futuro sean capaces de compartir nuestro entorno.

El trabajo presenta diferentes contribuciones en las áreas de control postural de robots humanoides, planificación de comportamientos, control no lineal, aprendizaje por demostración y aprendizaje por refuerzo. En primer lugar se desarrollan un conjunto de métodos y formulaciones matemáticas sobre los que se sustenta la tesis, describiendo conceptos de modelado de robots humanoides, generación de trayectorias de locomoción y generación de trayectorias del cuerpo completo. 

A continuación se estudia el proceso de aprendizaje humano, para desarrollar un novedoso método de transferencia de una tarea postural de un humano a un robot, usando como métrica de comparación el objetivo de la acción demostrada, que es codificada a través del refuerzo asociado a la ejecución de dicha tarea. 

Como evolución del trabajo anterior, se generaliza este proceso para la realización de un conjunto de comportamientos secuenciales, que son de nuevo realizados por el robot basándose en las demostraciones de un ser humano.

Seguidamente se estudia la ejecución de movimientos posturales utilizando un método de control robusto ante imprecisiones en el modelado del robot. 

Para finalizar, se presenta una arquitectura que aglutina los métodos de planificación y el control postural desarrollados en los capítulos anteriores. Esto se complementa con un módulo de reconocimiento del entorno y extracción del espacio libre para poder planificar y generar movimientos seguros en dicho entorno.

La justificación experimental de la tesis se ha desarrollado con el robot humanoide HOAP-3. En este robot se han implementado tareas como caminar, levantarse de una silla, bailar o abrir una puerta. Todo ello haciendo uso de las técnicas propuestas en este trabajo.


%


